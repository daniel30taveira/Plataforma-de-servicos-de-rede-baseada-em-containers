\chapter{Arquitetura e Implementação}
\label{chap:implementacao}

Este capítulo detalha o processo prático de implementação da plataforma de serviços de rede. A abordagem adotada seguiu as melhores práticas de "Infraestrutura como Código" (IaC), garantindo que todo o ambiente pudesse ser reconstruído de forma automatizada e previsível.

\section{Configuração do Ambiente Base}

O primeiro passo consistiu na preparação do \textit{host} físico que iria albergar os \textit{containers}. Para simular um ambiente de \textit{data center} isolado, mas acessível para testes, recorreu-se ao \textit{hypervisor} VMware para criar uma máquina virtual (VM) a correr o Ubuntu Server 24.04 LTS. 

A interface de rede da VM foi configurada em modo \textit{Bridged}, permitindo-lhe obter um endereço IP dedicado na rede local. A Figura \ref{fig:ubuntu_ip} demonstra o acesso inicial ao servidor por SSH e a verificação da sua configuração de rede.

\begin{figure}[h!]
    \centering
    \includegraphics[width=0.8\textwidth]{ubuntu_ip.png}
    \caption{Verificação do endereço IP e acesso SSH ao Ubuntu Server.}
    \label{fig:ubuntu_ip}
\end{figure}

O acesso remoto por SSH é crucial, uma vez que permite operar o servidor de forma \textit{headless} (sem interface gráfica), otimizando o consumo de recursos e preparando o terreno para a automação remota.

\section{Automação da Infraestrutura com Ansible}

Para cumprir o requisito de reprodutibilidade do projeto, optou-se por não realizar instalações manuais no servidor. Em vez disso, introduziu-se o \textbf{Ansible} como ferramenta de gestão de configuração. O Ansible liga-se ao servidor por SSH e executa instruções definidas em ficheiros YAML (\textit{Playbooks}), garantindo que o estado final do servidor é sempre o esperado.

O ponto de partida da automação é o ficheiro de inventário (\texttt{inventory.ini}), onde se define o endereço IP e o utilizador de acesso ao servidor de destino, conforme ilustrado na Figura \ref{fig:ansible_inventory}.

\begin{figure}[h!]
    \centering
    \includegraphics[width=0.8\textwidth]{ansible_inventory.png}
    \caption{Estrutura do repositório e ficheiro de inventário do Ansible.}
    \label{fig:ansible_inventory}
\end{figure}

\subsection*{Nota sobre o Ambiente de Execução (Control Node)}

É importante salientar uma particularidade da arquitetura de \textit{deploy}. Uma vez que a máquina física do administrador opera com o sistema operativo Windows, que não suporta nativamente a execução do Ansible, adotou-se o \textbf{WSL (Windows Subsystem for Linux)} como \textit{Control Node}. 

O WSL fornece um ambiente Linux completo e integrado, permitindo a instalação do Ansible e a execução transparente dos \textit{Playbooks}. Através desta abordagem, o terminal WSL comunica de forma segura via protocolo SSH com a máquina virtual VMware (\textit{Managed Node}), aplicando as configurações sem necessitar de ferramentas adicionais ou de dual-boot no sistema hospedeiro.

Com o inventário configurado, o passo seguinte consistiu na criação do \textit{Playbook} responsável por instalar as dependências base do sistema, nomeadamente o motor de \textit{containers} (Docker), que servirá de alicerce para todos os serviços subsequentes.

Para materializar a instalação do motor de \textit{containers}, desenvolveu-se o \textit{Playbook} \texttt{setup\_docker.yml}. Este \textit{script} declarativo é responsável por executar as seguintes tarefas no servidor de destino:

\begin{enumerate}
    \item Atualização da \textit{cache} dos repositórios APT.
    \item Instalação de pacotes utilitários e de certificados de segurança.
    \item Instalação do motor principal (\texttt{docker.io}) e do \textit{plugin} \texttt{docker-compose-v2}, fundamentais para a orquestração dos serviços da plataforma.
    \item Ativação do serviço \textit{daemon} do Docker no \texttt{systemd}, garantindo a sua inicialização automática em caso de reinício do servidor.
    \item Adição do utilizador remoto ao grupo \texttt{docker}, implementando o princípio de menor privilégio ao evitar o uso constante de credenciais \textit{root} para a gestão de \textit{containers}.
\end{enumerate}

A execução deste \textit{Playbook} a partir da máquina local do administrador garante que o servidor \textit{host} fica imediatamente pronto a receber os serviços da plataforma. A Figura \ref{fig:ansible_sucesso} ilustra o resultado (em jargão técnico, o \textit{Play Recap}) de uma execução bem-sucedida do Ansible contra a máquina virtual, demonstrando a idempotência da ferramenta (aplicação das configurações sem causar erros em execuções repetidas).

\begin{figure}[h!]
    \centering
    \includegraphics[width=0.9\textwidth]{ansible_sucesso.png}
    \caption{Execução bem-sucedida do \textit{Playbook} de instalação do Docker.}
    \label{fig:ansible_sucesso}
\end{figure}